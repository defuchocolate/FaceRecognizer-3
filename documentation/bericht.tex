%% Basierend auf einer TeXnicCenter-Vorlage von Tino Weinkauf.
%%%%%%%%%%%%%%%%%%%%%%%%%%%%%%%%%%%%%%%%%%%%%%%%%%%%%%%%%%%%%%

%%%%%%%%%%%%%%%%%%%%%%%%%%%%%%%%%%%%%%%%%%%%%%%%%%%%%%%%%%%%%
%% HEADER
%%%%%%%%%%%%%%%%%%%%%%%%%%%%%%%%%%%%%%%%%%%%%%%%%%%%%%%%%%%%%
\documentclass[a4paper,oneside,12pt]{article}
\usepackage{geometry}
\geometry{a4paper,left=40mm,right=30mm, top=25mm, bottom=25mm} 
% Alternative Optionen:
%	Papiergr��e: a4paper / a5paper / b5paper / letterpaper / legalpaper / executivepaper
% Duplex: oneside / twoside
% Grundlegende Fontgr��en: 10pt / 11pt / 12pt


%% Deutsche Anpassungen %%%%%%%%%%%%%%%%%%%%%%%%%%%%%%%%%%%%%
\usepackage[ngerman]{babel}
\usepackage[utf8]{inputenc}
\usepackage[T1]{fontenc}
\usepackage{lmodern} %Type1-Schriftart f�r nicht-englische Texte
\usepackage{float}
\usepackage{floatflt}

%% Packages f�r Grafiken & Abbildungen %%%%%%%%%%%%%%%%%%%%%%
\usepackage{graphicx} %%Zum Laden von Grafiken
%\usepackage{subfig} %%Teilabbildungen in einer Abbildung
%\usepackage{pst-all} %%PSTricks - nicht verwendbar mit pdfLaTeX

%% Beachten Sie:
%% Die Einbindung einer Grafik erfolgt mit \includegraphics{Dateiname}
%% bzw. �ber den Dialog im Einf�gen-Men�.
%% 
%% Im Modus "LaTeX => PDF" k�nnen Sie u.a. folgende Grafikformate verwenden:
%%   .jpg  .png  .pdf  .mps
%% 
%% In den Modi "LaTeX => DVI", "LaTeX => PS" und "LaTeX => PS => PDF"
%% k�nnen Sie u.a. folgende Grafikformate verwenden:
%%   .eps  .ps  .bmp  .pict  .pntg


%% Packages f�r Formeln %%%%%%%%%%%%%%%%%%%%%%%%%%%%%%%%%%%%%
\usepackage{amsmath}
\usepackage{amsthm}
\usepackage{amsfonts}


%% Zeilenabstand %%%%%%%%%%%%%%%%%%%%%%%%%%%%%%%%%%%%%%%%%%%%
%\usepackage{setspace}
%\singlespacing        %% 1-zeilig (Standard)
%\onehalfspacing       %% 1,5-zeilig
%\doublespacing        %% 2-zeilig


%% Farben %%%%%%%%%%%%%%%%%%%%%%%%%%%%%%%%%%%%%%%%%%%%%%%%%%%
\usepackage{color}
\usepackage{framed}
\usepackage{colortbl}
\definecolor{middlegray}{rgb}{0.5,0.5,0.5}
\definecolor{lightgray}{rgb}{0.8,0.8,0.8}
\definecolor{orange}{rgb}{0.8,0.3,0.3}
\definecolor{yac}{rgb}{0.6,0.6,0.1}
\definecolor{darkgray}{rgb}{0.3,0.3,0.3}
\definecolor{blue}{rgb}{0,0,1}
\definecolor{green}{rgb}{0,0.6,0}
\definecolor{yellow}{rgb}{1,1,0}
\definecolor{shadecolor}{gray}{.85}
\usepackage[table]{xcolor}
\newcommand{\tblue}{\cellcolor{blue!25}}
\newcommand{\tred}{\cellcolor{red!25}}
\newcommand{\tgreen}{\cellcolor{green!25}}
\newcommand{\tyellow}{\cellcolor{yellow!25}}


%% CodeListings %%%%%%%%%%%%%%%%%%%%%%%%%%%%%%%%%%%%%%%%%%%%%
\usepackage{listings}
\lstset{
	basicstyle=\scriptsize\ttfamily,
	keywordstyle=\bfseries\ttfamily\color{blue},
	breaklines=true,
	stringstyle=\color{darkgray}\ttfamily,
	commentstyle=\color{green}\ttfamily,
	emph={square}, 
	emphstyle=\color{blue}\texttt,
	emph={[2]root,base},
	emphstyle={[2]\color{yac}\texttt},
	showstringspaces=false,
	flexiblecolumns=false,
	captionpos=b,
	tabsize=2,
	numbers=left,
	numberstyle=\tiny,
	numberblanklines=true,
	stepnumber=1,
	numbersep=10pt,
	xleftmargin=15pt,
	frame=single
}

%% Verlinktes Inhaltsverzeichnis %%%%%%%%%%%%%%%%%%%%%%%%%%%
\usepackage[colorlinks,
pdfpagelabels,
pdfstartview = FitH,
bookmarksopen = true,
bookmarksnumbered = true,
linkcolor = black,
plainpages = false,
hypertexnames = false,
citecolor = black] {hyperref}


%% Sonstiges %%%%%%%%%%%%%%%%%%%%%%%%%%%%%%%%%%%%%%%%%%%%%%%
\usepackage{float}
\usepackage{tabularx}
\usepackage{amsmath}
\usepackage{subfigure}
\usepackage{multirow}
\usepackage{ulem}
\usepackage{wrapfig}

\setlength{\parindent}{0pt}


%% Andere Packages %%%%%%%%%%%%%%%%%%%%%%%%%%%%%%%%%%%%%%%%%%
%\usepackage{a4wide} %%Kleinere Seitenr�nder = mehr Text pro Zeile.
%\usepackage{fancyhdr} %%Fancy Kopf- und Fu�zeilen
%\usepackage{longtable} %%F�r Tabellen, die eine Seite �berschreiten


%% execute scripts %%%%%%%%%%%%%%%%%%%%%%%%%%%%%%%%%%%%%%%%%%
\newcommand{\visioToPDF}[1] {
	\immediate\write18{convert_scripts\string\\visioToPDF.bat #1}
}


%%%%%%%%%%%%%%%%%%%%%%%%%%%%%%%%%%%%%%%%%%%%%%%%%%%%%%%%%%%%%
%% Anmerkungen
%%%%%%%%%%%%%%%%%%%%%%%%%%%%%%%%%%%%%%%%%%%%%%%%%%%%%%%%%%%%%
%
% Zu erledigen:
% 1. Passen Sie die Packages und deren Optionen an (siehe oben).
% 2. Wenn Sie wollen, erstellen Sie eine BibTeX-Datei
%    (z.B. 'literatur.bib').
% 3. Happy TeXing!
%
%%%%%%%%%%%%%%%%%%%%%%%%%%%%%%%%%%%%%%%%%%%%%%%%%%%%%%%%%%%%%


%%%%%%%%%%%%%%%%%%%%%%%%%%%%%%%%%%%%%%%%%%%%%%%%%%%%%%%%%%%%%
%% Optionen / Modifikationen
%%%%%%%%%%%%%%%%%%%%%%%%%%%%%%%%%%%%%%%%%%%%%%%%%%%%%%%%%%%%%

%\input{optionen} %Eine Datei 'optionen.tex' wird hierf�r ben�tigt.
%% ==> TeXnicCenter liefert m�gliche Optionendateien
%% ==> im Vorlagenarchiv mit (Datei | Neu von Vorlage...).


%%%%%%%%%%%%%%%%%%%%%%%%%%%%%%%%%%%%%%%%%%%%%%%%%%%%%%%%%%%%%
%% DOKUMENT
%%%%%%%%%%%%%%%%%%%%%%%%%%%%%%%%%%%%%%%%%%%%%%%%%%%%%%%%%%%%%
\begin{document}
\thispagestyle{empty}

\title {
	\huge \textsc{TITEL}
}
	
\author {
	\begin{tabular}{rl}
		\large Matthias Böffel & \small Matrikel Nr.: 864483 \\ 
	\end{tabular}
}

\maketitle
\vfill
\begin{figure}[H]
\centering
\small Fachhochschule Kaiserslautern\\University of Applied Sciences\\
\bigskip
\large Betreuer: Prof. Dr. X\\
\bigskip
\includegraphics[scale=0.4]{images/fhlogo.jpg}
\end{figure}
\newpage
\tableofcontents
%\cleardoublepage %Das erste Kapitel soll auf einer ungeraden Seite beginnen.
%\pagestyle{plain} %%Ab hier die Kopf-/Fusszeilen: headings / fancy / ...

\newpage
\section*{Einleitung}
Text

%Thema und Zielsetzung: Stellen Sie zunächst Thema und Zielstellung der Arbeit vor.
%Theorie: Vermitteln Sie Ihre Theorie(n) über das Thema und geben Sie an, auf was sich Ihre Theorie stützt.
%Fragestellung: Teilen Sie mit, welche Fragen in der folgenden Arbeit beantwortet werden.
%Quellen: Welche Quellen haben Sie für Ihre Arbeit genutzt bzw. wie haben Sie Ihre Frage(n) beantwortet?
%Ergebnis: Führen Sie Ihre Ergebnisse auf, also teilen Sie mit, was Sie herausgefunden haben.
%Fazit: Stellen Sie am Ende des Abstracts eine Quintessenz auf. Sie können Ihr Fazit auch mit einer %Zukunftsprognose verbinden.


\newpage
\section{Hauptteil} \label{sec:hauptteil}
\subsection{Hauptteil - Part1}
Text

%\bigskip
%\begin{figure}[H]
	%\visioToPDF{images/datei.pdf}
	%\centering
	%\includegraphics[scale=0.7]{images/datei.pdf}
	%\caption{Beschriftung}
	%\label{fig:layer}
%\end{figure}
%\bigskip

%\lstset{language=Java}
%\begin{lstlisting}[caption=Listing, label=lst:Listing]
%\end{lstlisting}

%\begin{shaded}
%Text in Textbox
%\end{shaded}

%\ref{sec:hauptteil}
%\cite{audio_architecture}
\include{fazit}

\newpage

%% Literaturverzeichnis
%% ==> Eine Datei 'literatur.bib' wird hierf�r ben�tigt.
%% ==> Sie m�ssen hierf�r BibTeX verwenden (Projekt | Eigenschaften... | BibTeX)

\addcontentsline{toc}{section}{Literaturverzeichnis}
%\nocite{*} %Auch nicht-zitierte BibTeX-Eintr�ge werden angezeigt.
\bibliographystyle{alpha} %Art der Ausgabe: plain / apalike / amsalpha / ...
\bibliography{literatur} %Eine Datei 'literatur.bib' wird hierf�r ben�tigt.

%% Abbildungsverzeichnis
%\clearpage
%\addcontentsline{toc}{section}{Abbildungsverzeichnis}
%\listoffigures

%% Tabellenverzeichnis
%\clearpage
%\addcontentsline{toc}{section}{Tabellenverzeichnis}
%\listoftables


%%%%%%%%%%%%%%%%%%%%%%%%%%%%%%%%%%%%%%%%%%%%%%%%%%%%%%%%%%%%%
%% ANH�NGE
%%%%%%%%%%%%%%%%%%%%%%%%%%%%%%%%%%%%%%%%%%%%%%%%%%%%%%%%%%%%%
\appendix
%% ==> Schreiben Sie hier Ihren Text oder f�gen Sie externe Dateien ein.

%\input{Dateiname} %Eine Datei 'Dateiname.tex' wird hierf�r ben�tigt.

\end{document}

